\chapter{Introduction}
\label{c:intro}

Over the past several decades, the technology available for performing quantum physics experiments has advanced to a point where isolating, controlling, and measuring individual quantum degrees of freedom has become routine.  Early in this process, much attention was focused on creating techniques aimed at harnessing ``naturally-occurring'' quantum systems into this regime.  For example, precision measurement and manipulation of single electronic state transitions in the spectra of individual highly excited or ionized atoms has resulted in the direct observation of a number of fundamental quantum effects in experiments with a high degree of conceptual simplicity and beauty\footnote{The experimental apparatuses themselves, however, are of course far from simple.} \cite{PhysRevLett.65.976,PhysRevLett.72.3339,Nogues1999,Monroe24051996,Myatt2000,RevModPhys.73.565}, and were the subject of the 2012 Nobel prize in physics \cite{Nobelprize.org2012}.  More recently, the type and variety of natural quantum systems which can be manipulated at the single degree of freedom level have proliferated, including cold atomic gasses trapped in optical lattices \cite{Bloch2005}, single spins in diamond \cite{ISI:000321694300003}, the nuclear and electronic spins of a single phosphorous donor in silicon \cite{Pla2012}, and the spin and charge degrees of freedom of a single electron trapped in a quantum dot \cite{ISI:000321694300004} (this list is not exhaustive).

While these systems may not all be exactly ``natural'', per se, the quantum degrees of freedom are all fundamentally microscopic: single electron orbitals, the spins of single atoms or electrons, or the motional degree of freedom of a single atom.  Remarkably, there now exist classes of coherent systems whose quantum degrees of freedom are macroscopic quantities that have been specially \textit{engineered} to express quantum behavior.  In these systems, the collective motion of a very large number of constituent particles constitute a single quantum variable, with all of the same resulting richness and perplexing implications of the quantum systems created by nature.  Unlike natural systems, however, these engineered quantum systems can re-arranged, twisted into different shapes, mutated, and moved into obscure and interesting nooks of their parameter spaces, in some cases permitting the observation of quantum phenomena which are difficult or impossible to observe in hitherto explored systems.

\section{Superconducting qubits and circuit quantum electrodynamics}

A rigorous proposition for searching for quantum-coherent behavior in a macroscopic object came from Leggett in 1980 \cite{1980PThPS..69...80L}; in this paper, he presciently observed that superconducting circuits were the most promising platform then known for observing such behavior.  Indeed, macroscopic quantum tunneling was observed in the phase difference across a current-biased Josephson junction at UC Berkeley in 1987 \cite{PhysRevB.35.4682}.  This experiment set the stage for the full consideration of the currents and voltages in superconducting circuits as quantum degrees of freedom, eventually leading to the first observation of a coherent macroscopic superposition state in a superconducting circuit called a Cooper pair box at NEC in 1999 \cite{Nakamura1999}---the first demonstration of a \textit{superconducting quantum bit (qubit)}.  This first device only maintained quantum coherence over a timescale of about 1 ns.  Sixteen years later, as of the writing of this thesis, superconducting qubits routinely achieve coherence times of tens of microseconds \cite{Paik_3DT,Chang2013}, an improvement of more than four orders of magnitude, and there is still no generally agreed-upon fundamental upper limit to these coherence times.

Superconducting qubits came into their own as a platform for testing quantum theory, especially quantum measurement and feedback control, with the marriage of cavity quantum electrodynamics \cite{RevModPhys.73.565} and superconducting circuits, resulting in the new paradigm of \textit{circuit QED} (or \textit{cQED}) in pioneering experiments at Yale \cite{cQEDtheory,Wallraff2004}.  The cQED architecture can provide controllable strong coupling between a single microwave-frequency photon and a superconducting qubit, as well as ideal, non-destructive measurement of the quantum state of the qubit \cite{Vijay:2011kx,slichterthesis} or the photon state of the cavity \cite{Schuster2007}.  These properties permit, for example, the controlled generation of exotic quantum optical states \cite{Hofheinz2009,Vlastakis2013}, the monitoring and reconstruction of the quantum trajectory of a qubit undergoing continuous measurement \cite{murch_observing_2013,Weber2014,Weber2014a}, and the first observation of the enhancement of an atomic lifetime using squeezed light \cite{murch_reduction_2013}.  The experiment which is the subject of the first half of this thesis, the implementation of a quantum feedback control protocol, crucially relies on the exquisite quantum measurement capability provided by cQED.

Besides their applicability to the study of quantum mechanics, superconducting qubits and cQED have emerged as a viable platform on which to implement fault-tolerant quantum computing protocols \cite{Nielsen2000}, such as the ``surface code'' protocol \cite{PhysRevA.86.032324}.  In the last two years, error rates in multi-qubit devices have approached the minimum threshold necessary to implement the surface code error correction procedures \cite{Barends2014,Riste2014,Chow2014}.  Many technological challenges remain between these proof-of-principle experiments and a viable quantum computer.  At present, high-quality qubit state measurement---essential for the surface code---is performed in these systems with the aid of ultra-low-noise microwave amplifiers based on Josephson junctions \cite{Castellanos-Beltran2008,JPCNature,Hatridge:2011zr}.  The performance of these amplifiers is one of many obstacles which stand in the way of realizing a large-scale quantum computer based on superconducting qubits \cite{Mutus2014,Kelly2015}; the device which constitutes the subject of the second half of this thesis directly addresses this need.

\section{Quantum feedback control}

Feedback control schemes are ubiquitous in classical systems for stabilizing the state of that system against disturbances.  Thermostats, anti-lock brakes, pacemakers, and aircraft flight control systems all utilize the outcome of a measurement to automatically and autonomously steer a system towards a desired state, and hold it there even in the presence of fluctuations.  The operation of these feedback protocols is predicated on the idea that making a measurement of the state of the system need not alter that state.  Of course, if we apply feedback control to a quantum system, this predicate no longer holds: measurements in quantum mechanics are fundamentally invasive \cite{qmcontrol_book}. Thus, quantum feedback control is faced with an additional fundamental challenge: how can we hope to stabilize the state of a quantum system using feedback control if our very measurement of that state disturbs it?

As in many areas of physics, theoretical development in the field of quantum control has significantly outpaced experimental demonstration.  This is in large part due to the technological challenges associated with realizing controllable quantum systems discussed so far in this chapter.  Furthermore, in order to address the question at the end of the previous paragraph, we require not just a controllable quantum system, but one which is capable of realizing a \textit{nearly-ideal, minimally-invasive, continuous quantum measurement}.  The textbook picture of the quantum measurement process involves the instantaneous projection of the quantum system into one of its eigenstates; if we desire to utilize a feedback protocol to stabilize, say, a superposition state of that system, then such a projective measurement is in a sense maximally invasive.  We require, instead, a measurement platform where we can slow down the time scale associated with projecting the system into an eigenstate, until the measurement is slower than the time scale over which feedback occurrs.  Then, we can use our knowledge of the measurement outcome and the action of the feedback loop to \textit{undo the back-action of the measurement} as well as correct for the effect of external disturbances on the system.

The first demonstration of closed-loop feedback control of a single quantum system was implemented by the cavity QED team at ENS in 2011 \cite{haroche_fb}.  In this beautiful experiment, the photon number state of a microwave cavity is weakly probed by a stream of atoms prepared in highly excited Rydberg states. Each atom makes a very weak, non-destructive measurement of the photon number, such that about 50-100 atoms must be detected to fully determine it.  Using a complex Bayesian model, a real-time computer applies a classical control correction to the cavity field after the detection of each atom, with the target of stabilizing the cavity into a particular highly-nonclassical definite photon number state (aka a Fock state).  The action of the feedback control quickly projects the cavity into the desired Fock state with high fidelity, and successfully stabilizes this state in the presence of decoherence associated with the cavity decay lifetime.

Circuit QED provides another path towards realizing a faithful, weak quantum measurement: the continuous weak measurement of the state of the qubit by the small phase shift it induces in the photon field in the cavity \cite{Boissonneault2009,koro11,Girvin2014}.  By measuring the field escaping from the cavity, a continuous tracking of the qubit state is possible, provided that all of the information carried by this field is faithfully recovered by the experimental apparatus.  The experiment which comprises the first half of the results in this thesis leverages this continuous, weak measurement of the qubit state to perform a feedback stabilization of the coherent Rabi oscillations of a superconducting qubit.  The feedback protocol not only corrects for external decoherence of the qubit, but also self-corrects the stochastic back-action of the measurement process itself.

The power carried by the microwave field which conveys qubit state information away from the cavity is extremely small, on the order of a few femtowatts. In contrast, the electronics used to process this signal at room temperature typically expect milliwatts.  As such, to realize this experiment, we require an ultra-low-noise preamplifier which is itself governed by (and ideally limited by) quantum mechanics.

\section{Josephson parametric amplifiers}

In order to realize an amplifier whose performance is limited by quantum mechanics, we turn to the class of devices known as \textit{parametric amplifiers}.  The operating principle of a parametric amplifier is based on the harmonic modulation of a parameter of a nonlinear dynamical system by a strong drive called the \textit{pump}.  The nonlinear element in the system provides one or more terms in the dynamical equation which couple energy between oscillations at different frequencies; thus, when arranged correctly, a small initial signal oscillation grows over time through the nonlinear interaction with the pump, realizing gain.  The mechanism of parametric amplification does not necessarily mandate any energy dissipation; a lack of dissipation also implies a lack of additional fluctuations, implying that parametric amplifiers have the potential to realize quantum-limited noise performance \cite{clerk_revmod}.

The Josephson tunnel junction is a circuit element unique to superconducting circuits which is both highly nonlinear and non-dissipative.  Parametric amplifiers utilizing the Josephson nonlinearity were first demonstrated in 1975 \cite{Feldman1975}, and development continued through the 80s and 90s \cite{1060904,1063665,Yurke1989,Yurke:1996ys}, though the quantum limit proved elusive and these devices did not see much practical use.  More recently, the promising applications of superconducting devices in the fields of quantum measurement and quantum information reignited interest in this type of amplifier, and improved designs have demonstrated robustly quantum-limited noise performance with high gain and sufficient bandwidth for many applications \cite{Castellanos-Beltran2008,JPCNature,Hatridge:2011zr}.

These Josephson parametric amplifiers (JPAs) have been integral to realizing high-quality measurements of superconducting qubits and have enabled many of the groundbreaking experiments mentioned so far in this chapter.  The development of this type of amplifier has been a central activity at the Quantum Nanoelectronics Laboratory (QNL) at UC Berkeley, where the experiments described in this thesis took place.  Improvements in JPA design at QNL have led to the observation of quantum jumps in a superconducting qubit \cite{Vijay:2011kx,slichterthesis}, the demonstration of high-fidelity qubit readout and heralded state preparation \cite{fluxqb}, the observation and statistical analysis of quantum trajectories \cite{murch_observing_2013,Weber2014,Weber2014a}, and the demonstration of measurement-induced entanglement generation between remote qubits \cite{Roch2014}.

The circuit topology for essentially all JPAs involves embedding one or more junctions in some kind of resonant circuit, consisting of one or more distributed-element transmission line resonators or lumped-element LC resonators, with the nonlinear Josephson inductance contributing some fraction of the total inductance.  Although a variety of design improvements have expanded the performance of these devices, the resonant topology fundamentally introduces a limitation called the \textit{gain-bandwidth product}: the product of the gain in amplitude units and the bandwidth is limited to a constant, typically 10 MHz to 1 GHz or so.  As such, essentially all JPAs with amplitude gain of 10 are limited to bandwidths on the order of 1-100 MHz, with most devices achieving about 10 MHz.  Furthermore, the dynamics of the nonlinear resonator restrict how much power can be used to pump the devices, which limits the amount of signal power a JPA can faithfully amplify.

For many experiments this performance is sufficient; however, for the construction of a large-scale quantum computer based on superconducting circuits, it would be desirable to have an amplifier with a much larger bandwidth, which is capable of handling multiple input signals simultaneously, while still retaining quantum-limited noise performance.  The design and realization of a JPA based on a fundamentally different circuit topology---a nonlinear transmission line, rather than a nonlinear resonator---comprises the second half of the results described in this thesis.  The resulting device, dubbed the Josephson traveling-wave parametric amplifier (JTWPA), is not saddled with the same fundamental limitations as a resonator-based JPA.  The JTWPA is able to achieve large gain over gigahertz-scale bandwidths, faithfully amplifying signal powers an order of magnitude larger than the best JPAs yet demonstrated, while achieving nearly quantum-limited noise performance.

\section{Thesis overview}

I begin chapter \ref{c:qm} with the aim of providing an intuitive picture of how quantum limits on measurement arise by discussing two classic thought experiments.  Most of the remainder of the chapter is dedicated to the description of various classes of quantum measurements, focusing on those most relevant to the experiments performed in this thesis, including a brief description of the quantum Bayesian approach to quantum measurement.  I conclude the chapter with a discussion of the theory of the quantum control protocol implemented in the Rabi stabilization experiment, including the analytical derivation of the performance of the feedback loop.

In chapter \ref{c:scqb}, I give an overview of how to construct a qubit from superconducting circuits.  Rather than following the historical approach of starting with the Cooper pair box, I take an alternative approach which I find more intuitive: because modern superconducting qubits are essentially weakly-anharmonic oscillators, I start with a general discussion of anharmonic oscillators and then describe how such a device can be realized using superconducting circuit elements.  Finally, I introduce the basics of cavity QED and the circuit QED implementation, specifically addressing the parameter regime applicable to weak quantum measurement.  Chapter \ref{c:paramps} is the third and final chapter of background material, comprising a discussion of quantum-limited amplifers.  I begin the chapter with a general discussion of the origin of quantum limits on amplification, followed by a description of JPAs, their performance, and the limitations to their performance.  These limitations lead naturally to a brief discussion of traveling-wave amplifiers and the JTWPA as an alternative to the traditional JPA.

Chapters \ref{c:qfb} and \ref{c:qfb_results} comprise the experimental realization of the stabilization of Rabi oscillations using quantum feedback control.  In chapter \ref{c:qfb}, I provide a thorough description of the experimental apparatus, including a variety of important calibration and tuning experiments necessary to demonstrate high-quality feedback and interpret the results.  Chapter \ref{c:qfb_results} is entirely dedicated to the experimental results attained in the quantum feedback experiment, including frequency- and time-domain measurements of the stabilized state as well as tomographic state reconstruction and validation.  In addition to the results directly pertaining to feedback stabilization of Rabi oscillations, this chapter includes a beautiful piece of unpublished data related to variable-strength continuous quantum measurement.

Chapters \ref{c:twpa_theory} and \ref{c:twpa_exp} comprise the theory and experimental results for the JTWPA, respectively.  In chapter \ref{c:twpa_theory}, I provide some background on nonlinear optics, as the theory of the JTWPA is partially derived from it.  I continue with a presentation of the derivation of the continuum wave equation (the key link between the circuit-theory description of the JTWPA and the nonlinear-optical description) and the requirements for realizing efficient parametric amplification.  Next, I discuss the derivation of the dispersion relation of the JTWPA and the associated ``resonant phase matching'' dispersion-engineering technique we created to satisfy the crucial nonlinear optical phase matching criterion needed for efficient amplification.  I close the chapter with a detailed presentation of the theoretical performance predicted for a practical device.

In chapter \ref{c:twpa_exp}, I describe a precise experimental assessment of a JTWPA device, validating the theory presented in chapter \ref{c:twpa_theory}.  The amplifier calibration experiments utilize a cQED system in the weak measurement limit to enable high-precision noise measurements.  I conclude the chapter with a discussion of the performance of the JTWPA in a projective qubit measurement, and extrapolate those results to show that it could serve as the low-noise preamplifier for the simultaneous readout of as many as 20 superconducting qubits.  I close the thesis with some future directions in quantum feedback control and JTWPA development in chapter \ref{c:conclusion}.

\section{Summary of key results}

The quantum feedback experiment presented in this thesis is the first realization of a quantum feedback control protocol in a solid-state system, and also the first demonstration of a quantum feedback protocol for stabilizing a dynamical quantum state.  The protocol stabilizes the target state with an efficiency of about 45\%, in excellent agreement with the predictions from theory when the imperfections in the experiment are accounted for.  As such, this experiment constitutes a validation of a significant body of quantum feedback control theory.  To quote Howard Wiseman in his \textit{Nature} News \& Views article on the experiment, ``With [this] experiment, solid-state physics joins quantum optics at the forefront of quantum feedback-control investigations.'' \cite{Wiseman2012}

The performance of the JTWPA device described in this thesis represents an improvement on the state of the art by more than an order of magnitude in both bandwidth and signal power handling, with a clear road towards further improvements in these metrics.  The noise performance of the amplifier is essentially quantum-limited, providing a system noise temperature comparable to those achieved with the lowest-noise JPAs yet demonstrated.  Furthermore, the agreement with theory predictions is remarkably close considering the significant complexity and nonlinearity of the device.  This agreement implies that further design revisions and device engineering can be reliably conducted on the basis of the existing theory, with the expectation that new device designs should perform as expected.  This type of amplifier will likely prove to be a key component in near-term demonstrations of small-scale quantum computers based on superconducting qubits.